\chapter{Reflection and outlook}
The previous chapter reached the conclusion that machine learning algorithms are, in fact, feasible for detecting collision detection in the case of the Sensorboard. This shows the potential that machine learning algorithms might have for detecting patterns corresponding to driving behaviours or accidents.
\\
Regardless of the positive result, this study has to be viewed critically. The process of building models and evaluating them is based on datasets generated in experiments in isolated conditions. As no real driving data was recorded in the dataset, the outcome of the research might not be completely transferable to a real driving scenario. Conditions like rough or irregular surfaces and unexpected driving manoeuvres will influence the data quality and therefore the model consistency. 
\\
Furthermore the results of this study can not be applied to other driving scenarios as patterns in data will differ based on the vehicle and might not be recognized as easily or with the same methods as described in this paper.
\\
Nevertheless the results of the research clearly indicate potential of this approach for similar use cases. The knowledge gained regarding the feasibility and best methods for a machine learning approach to this use case points further research in the right direction.
\\
In further research the results of this paper can be refined and checked by using sensor data originating from real driving of the Sensorboard as it resembles more realistic driving behaviour. This way the result can be applied to more realistic conditions. 
\\
In addition the experimental results of this study can be further examined. It can be studied why some algorithms perform better than others in the given scenario. The fact that tree based algorithms seem to work better here might be explained by the non linearity of the models created. In contrast the linear SVC and logistic regression represent generate linear models. The superiority of decision tree assembles like  the Random forest and gradient-boosted tree over a single decision tree might be explained by a reduction in variance of a classifier and thereby the prevention of overfitting and better generalization of the models. A definite explanation for those phenomenon can be explored in further research.
\\
Furthermore the knowledge gained during this research can transposed on other similar problems. Other vehicles like bikes, motorbikes or cars might as well allow for a machine learning based detection of events like collisions. 
\\
Even when applied only to a showcase scenario as in this paper, it becomes clear that machine learning applied to IoT data has a lot of potential. Being able to showcase this fact to clients will lead them to a smarter way of making use of sensor data in the information age. In this way the growing field of the Internet of Things in combination with machine learning might have an even bigger impact on industries and consumers.
\\
Rolling Window?
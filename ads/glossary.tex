%!TEX root = ../dokumentation.tex

%
% vorher in Konsole folgendes aufrufen:
%	makeglossaries makeglossaries dokumentation.acn && makeglossaries dokumentation.glo
%

%
% Glossareintraege --> referenz, name, beschreibung
% Aufruf mit \gls{...}
%
\newglossaryentry{Socket}{name={Socket},plural={Sockets},description={Ein Socket ist ein Kommunikationsendpunkt, der vom Betriebsystem bereitgestellt wird}}

\newglossaryentry{Hub}{name={Hub},plural={Hubs},description={Ein Hub bezeichnet einen Knotenpunkt in einem Netzwerk, welcher die Geräte sternförmig verbindet}}

\newglossaryentry{RFC Dokument}{name={RFC Dokument},plural={RFC Dokumente},description={Request for Comments Dokumente werden von der Internet Engineering Task Force genutzt um Internetprotokollstandards zu definieren oder Verfahren zu beschreiben}}

\newglossaryentry{Message Oriented Middleware}{name={Message Oriented Middleware},plural={Message Oriented Middleware},description={Die Message Oriented Middleware ist die Diensteschicht die auf der Übertragung von Nachrichten beruht}}

\newglossaryentry{Plug 'n Play}{name={Plug 'n Play},plural={Plug 'n Play},description={Plug 'n Play beschreibt ein Konzept bei dem Geräte an ein System angeschlossen werden, ohne Gerätetreiber installieren zu müssen oder Einstellungen vorzunehmen}}

\newglossaryentry{WebSphere MQ}{name={WebSphere MQ},plural={WebSphere MQ},description={WebSphere MQ ist Teil der WebSphere Produktlinie von IBM. Bei dem Produkt handelt es sich um eine Message Oriented Middleware}}

\newglossaryentry{International Conference on Information Networking}{name={International Conference on Information Networking},plural={International Conference on Information Networking},description={Die International Conference on Information Networking ist eine internationale wissenschaftliche Konferenz, bei der es um Informationsverarbeitung geht}}

\newglossaryentry{eingebettete Systeme}{name={Eingebettetes System},plural={Eingebettete Systeme},description={Ein Eingebettetes System bezeichnet ein elektronischen Rechner bzw. Computer, welcher in einem technischen Kontext eingebunden ist}}

\newglossaryentry{Bluetooth}{name={Bluetooth},plural={Bluetooth},description={Bluetooth ist ein drahtloser Netzwerkstandard bei dem die Geräte über kurze Instanz miteinander kommunizieren oder Daten übertragen}}

\newglossaryentry{Ethernet}{name={Ethernet},plural={Ethernet},description={Das Ethernet ist eine Netzwerktechnologie, welche Soft- und Hardware für kabelgebundene Netzwerke spezifiziert}}

\newglossaryentry{Multithreading}{name={Multithreading},plural={Multithreading},description={Als Multithreading bezeichnet man das gleichzeitige abarbeiten mehrerer Theads während eines Tasks oder Prozesses}}

\newglossaryentry{Watchdog}{name={Watchdog},plural={Watchdog},description={Als Watchdog wird eine Komponente eines Systems bezeichnet, welche die Funktion der anderen Komponenten überwacht und auf Fehler überprüft}}

\newglossaryentry{Ressourcenpools}{name={Ressourcenpools},plural={Ressourcenpools},description={Ressourcenpools sind geteilte Zugriffsmöglichkeiten auf Informationen oder Rechenleistung}}

\newglossaryentry{Big Data}{name={Big Data},plural={Big Data},description={Big Data bezeichnet Datenmengen, welche zu groß und zu komplex sind um sie mit klassischer Datenverarbeitung auswerten zu können}}

\newglossaryentry{TCP/IP}{name={TCP/IP},plural={TCP/IP},description={Transmission Control Protocol / Internet Protocol (TCP/IP) ist eine Netzwerkprotokollfamilie, welche wegen ihrer großen Bedeutung für das Internet auch als Internetprotokollfamilie bezeichnet wird}}

\newglossaryentry{Footprint}{name={Footprints},plural={Footprint},description={
Unter dem Footprint versteht man den Platz den eine Anwendung im Arbeitsspeicher einnimmt}}

\newglossaryentry{Handshake}{name={Handshake},plural={Handshakes},description={Der Handshake ist eine Bezeichnung für eine Steuerung des Datenflusses über die serielle Schnittstelle}}

\newglossaryentry{Websockets}{name={Websocket-Protokoll},plural={Websocket-Protokolle},description={Das WebSocket-Protokoll ist ein Netzwerkprotokoll, welches auf TCP basierend. Mit dem Protokoll wird eine bidirektionale Verbindung zwischen einer Webanwendung und einem Webserver aufgebaut}}

\newglossaryentry{Arduino}{name={Arduino},plural={Arduinos},description={Arduino ist eine Physical-Computing-Plattform. Der Hardwareteil besteht aus einem E/A-Board mit einem Mikrocontroller und analogen und digitalen Ein- und Ausgängen}}

\newglossaryentry{BidCoS}{name={BidCoS},plural={BidCoS},description={BidCoS ist der Funkstandard welchens das SmartHome-System HomeMatic von EQ-3 nutzt}}

\newglossaryentry{ZigBee}{name={ZigBee},plural={ZigBee},description={ZigBee ist ein Netzwerkstandart für drahtlose Netzwerke mit geringem Datenaufkommen}}

\newglossaryentry{RESTful Protokoll}{name={RESTful Protokoll},plural={RESTful Protokoll},description={Ein RESTful Protokoll, ist ein Protokoll welches auf dem Representational State Transfer Programmierparadigma basiert}}

\newglossaryentry{PreSharedKey}{name={Pre-Shared Key},plural={Pre-Shared Key},description={Pre-Shared Key ist ein symmetrische Verschlüsselungsverfahren bei den die Schlüssel vor dem Datenaustausch beiden Teilnehmern bekannt sein müssen}}

\newglossaryentry{IP-Multicast}{name={IP-Multicasting},plural={IP-Multicasting},description={IP-Multicasting ermöglicht es, in IP-Netzwerken effizient Pakete an viele Empfänger zur gleichen Zeit zu senden}}

